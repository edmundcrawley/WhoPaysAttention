% !TeX spellcheck = en_GB
\documentclass[12pt,pdftex,letterpaper]{article}
%            \usepackage{setspace}
\usepackage[dvips,]{graphicx} %draft option suppresses graphics dvi display
%            \usepackage{lscape}
%            \usepackage{latexsym}
%            \usepackage{endnotes}
%            \usepackage{epsfig}
%           \singlespace
\setlength{\textwidth}{6.5in}
\setlength{\textheight}{9in}
\addtolength{\topmargin}{-\topmargin} 
\setlength{\oddsidemargin}{0in}
\setlength{\evensidemargin}{0in}
\addtolength{\headsep}{-\headsep}
\addtolength{\topskip}{-\topskip}
\addtolength{\headheight}{-\headheight}
\setcounter{secnumdepth}{2}
%            \renewcommand{\thesection}{\arabic{section}}
% \renewcommand{\footnote}{\endnote}
\newtheorem{proposition}{Proposition}
\newtheorem{definition}{Definition}
\newtheorem{lemma}{lemma}
\newtheorem{corollary}{Corollary}
\newtheorem{assumption}{Assumption}
\newcommand{\Prob}{\operatorname{Prob}}
\clubpenalty 5000
\widowpenalty 5000
\renewcommand{\baselinestretch}{1.25}
\usepackage{amsmath}
\usepackage{amsthm}
\usepackage{amsfonts}
\usepackage{amssymb}
\usepackage{bbm}
\newcommand{\N}{\mathbb{N}}
\newcommand{\R}{\mathbb{R}}
\newcommand{\E}{\mathbb{E}}

\begin{document}
	
\section{Who Pays Attention - basic idea}

Consider a simple two-period optimization problem:

\begin{align}
\max u(C_1) + \beta u(C_2)
\end{align}
subject to:
\begin{align}
C_1 + \frac{1}{R} C_2 \leq Y_1 + \frac{1}{R} Y_2
\end{align}

The Euler equation is:
\begin{align}
u'(C_1) = \beta R u'(C_2) \\
\end{align}
Assuming log utility and linearizing:
\begin{align}
C_1 = \frac{1}{\beta R} C_2 \\
\end{align}
Plugging into budget eq:
\begin{align}
\frac{1}{R}(\frac{1}{\beta} + 1 )C_2 = Y_1 + \frac{1}{R} Y_2 \\
C_2 = \frac{RY_1 +  Y_2}{\frac{1}{\beta} + 1 } \\
C_1 = \frac{Y_1 + \frac{1}{R} Y_2}{1 + \beta }
\end{align}
\begin{align}
\frac{1}{R}(\frac{1}{\beta} + 1 )C_2 = Y_1 + \frac{1}{R} Y_2 \\
\log(C_2) = \log( \frac{RY_1 +  Y_2}{\frac{1}{\beta} + 1 } ) \\
C_1 = \frac{Y_1 + \frac{1}{R} Y_2}{1 + \beta }
\end{align}
Suppose, for simplicity, $\beta = 1$, $Y=Y_1=Y_2$, $R= 1+r$ with $r$ small such that $1/R \approx = 1-r$, then
\begin{align}
C_2 = (1+\frac{r}{2}) Y \\
C_1 = (1-\frac{r}{2 }) Y
\end{align}
Suppose you didn't pay attention to the change in $R$, then you would consume $C_1=C_2=Y$.\\
Loss of utility would be second order.

\newpage

Now assume you start owing a debt of 1, face value 1, in period 2, with an offsetting income of 1 next period. You have the option to refinance.\\
If $R$ goes up, you will not refinance - problem is identical to the above:
 \begin{align}
 C_2 = (1+\frac{r}{2}) Y \\
 C_1 = (1-\frac{r}{2 }) Y
 \end{align}
However, if $R$ goes down, you can refinance and only pay (1+r) next period
 \begin{align}
C_2 =  Y \\
C_1 = (1-r) Y
\end{align}
If you didn't notice this, loss to utility would be first order!!!

\section{A Two-Period Sticky Price Model}


\end{document}