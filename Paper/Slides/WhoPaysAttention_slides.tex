\input{./econtexRoot}\input{\econtexRoot/econtexPaths.tex}
\documentclass{beamer}
\usepackage{etoolbox}
\usepackage{comment}
\usepackage{graphicx}
%\usepackage{dtklogos}
\usepackage{dsfont}
\usepackage{amsmath,amssymb}
\usepackage{\econtexShortcuts}
\usepackage[english]{babel}
\usepackage[svgnames,hyperref]{}
\usepackage{empheq}
\usepackage[many]{tcolorbox}
\usepackage{remreset}
\usepackage{tikz} 
\usetikzlibrary{tikzmark,fit,shapes.geometric}
\usetikzlibrary{tikzmark,calc,arrows,shapes,decorations.pathreplacing}
\tikzset{every picture/.style={remember picture}}
\usepackage{cancel}
\usepackage{booktabs,natbib}
\setbeamercovered{invisible}

\usepackage{dcolumn}

\tcbset{highlight math style={enhanced,
		colframe=red!60!black,colback=yellow!50!white,arc=4pt,boxrule=1pt,
}}


\makeatletter
\@removefromreset{subsection}{section}
\patchcmd{\beamer@part}{\setcounter{subsection}{0}}{}{}
\makeatother
\setcounter{subsection}{1}
\setbeamercovered{transparent}
\setbeamertemplate{navigation symbols}{}%remove navigation symbols
\begin{comment}
\setbeamertemplate{footline}
{
	\hbox{\begin{beamercolorbox}[wd=1\paperwidth,ht=2.25ex,dp=1ex,right]{framenumber}%
			\usebeamerfont{framenumber}\insertframenumber{} \hspace*{2ex}
	\end{beamercolorbox}}%
	\vskip0pt%
}
\end{comment}


\mode<presentation>{}
%% preamble
\title[Who Pays Attention to Euler?]{Who Pays Attention to Euler?}
\author{Edmund Crawley}
%\date[03/05/2020]{March, 2020}
\date[03/05/2020]{}
\usetheme{Frankfurt}

\setbeamertemplate{navigation symbols}{}
\makeatletter
\setbeamertemplate{footline}
{%
	\hbox{\begin{beamercolorbox}[wd=1\paperwidth,ht=2.25ex,dp=1ex,right]{framenumber}%
		\usebeamerfont{framenumber}\insertframenumber{} \hspace*{2ex}
	\end{beamercolorbox}}%
	\vskip0pt%
	\pgfuseshading{beamer@barshade}%
	\ifbeamer@sb@subsection%
	\vskip-9.75ex%
	\else%
	\vskip-7ex%
	\fi%
	\begin{beamercolorbox}[ignorebg,ht=2.25ex,dp=3.75ex]{section in head/foot}
		\insertnavigation{\paperwidth}
	\end{beamercolorbox}%
	\ifbeamer@sb@subsection%
	\begin{beamercolorbox}[ignorebg,ht=2.125ex,dp=1.125ex,%
		leftskip=.3cm,rightskip=.3cm plus1fil]{subsection in head/foot}
		\usebeamerfont{subsection in head/foot}\insertsubsectionhead \insertframenumber{} \hspace*{2ex}
	\end{beamercolorbox}%
	\fi%
}%
\setbeamertemplate{headline}{%
}
\makeatother
\begin{document}
\setbeamertemplate{caption}{\raggedright\insertcaption\par}
\newcolumntype{d}[1]{D{.}{.}{#1}}
%circled draws a circle around a number
\newcommand*\circled[1]{\tikz[baseline=(char.base)]{
		\node[shape=circle,draw,inner sep=2pt] (char) {#1};}}

\begin{frame}[plain]
\titlepage
\end{frame}
\addtocounter{framenumber}{-1}
\section{Introduction}
\setbeamercovered{invisible}
\frame
{
	\frametitle{Interest Rates: For Whom is Inattention Costly?}
	Main Idea:
	\begin{itemize}
		\item Entirely rational for unconstrained households to ignore interest rates (second order)
		\item Constrained agents \textit{cannot} ignore interest rates: they directly determine constraints
		\item Refincing decisions are not ignored: they are first order
	\end{itemize}
	\pause
	\bigskip
	I examine a Two Agent New Keynesian model in which
	\begin{itemize}
		\item Unconstrained agents are inattentive
		\item Constrained agents are attentive
		\item Add refinancing a la \cite{greenwald_mortgage_2018}
	\end{itemize}
}
\frame{
\frametitle{Implications}
Puzzles resolved:
\begin{itemize}
	\item No Forward Guidance Puzzle
	\item Fed has control on long term real rates
	\item Hump shaped consumption response
\end{itemize}
\bigskip
Policy Implications:
\begin{itemize}
	\item Monetary Policy acts through redistribution (and investment)
	\item Much closer relation to fiscal policy
	\item Need to think through all implications
\end{itemize}
}
\frame{
\frametitle{Relation to Literature}
\begin{itemize}
\item \cite{wong_population_2016}, \cite{berger_prepayment_2018}, \cite{eichenbaum_refinance_2018}. Partial Equilibrium $\implies$ mortgages play an important role in monetary policy. Rely on long-term real rate changes.
\item \cite{greenwald_mortgage_2018}, \cite{garriga_monk_2019}. General equilibrium New Keynesian $\implies$ Mortgages do not play a role, long real rates don't move.
\item Rational Inattention literature $\implies$ hump shape responses BUT no heterogeneity in inattention.
\item Attention to refincing, Inattention to intertemporal substitution, can resolve these tensions in the literature.
\end{itemize}
}
\frame{
\frametitle{Evidence for Consumption Intertemporal Substitution}
\begin{itemize}
\item Macro: Complete failure of relation between real rates and consumption growth
\item Micro: No convincing evidence households respond to interest rate incentives
\item Sheer size of real interest rate movements: 30 year treasury down almost 2 percentage points since Nov 2018 $\implies$ I should increase consumption byover 10\% today (all else equal)
\item Evidence from asking financial advisors: when asked interest rates change their saving advice, they look at me like I'm crazy!
\item Evidence from default pension saving - people really don't pay attention to this decision! \cite{madrian_power_2001}
\end{itemize}
}
\section{Costs of Inattention}
\frame{
\frametitle{Costs of Inattention: A Two-Period Example}
Consider a two period model with consumer maximizing:
\begin{align*}
\log(C_1) +  \log(C_2)
\end{align*}
$R=1$ and income $Y_1 = Y_2 = Y$. Solution is $C_1 = C_2 = Y$
\pause
\bigskip
Now $R$ is increased to $1+r$. (Linearized) Solution is:
\begin{align*}
C_1 = (1-\frac{r}{2  }) Y\\
C_2 = (1+\frac{r}{2 }) Y 
\end{align*}
\pause
\bigskip
Suppose you didn't pay attention and consumed $C_1 = C_2 = Y$ as before. \textbf{Loss of utility would be second order.}
}
\frame{
\frametitle{Costs of Inattention: An Example with Refinancing}
\begin{itemize}
\item Now assume you start owing a debt of $D$ in period 2, with an offsetting income of $D$ in period 2.\\
\item You have the option to refinance at a face value of $D$.\\
\item Suppose debt is equal to income, $D=Y$\\
\end{itemize}
If $R$ goes up, you will not refinance - problem is identical to the above:
 \begin{align*}
 C_1 = (1-\frac{r}{2 }) Y\\
 C_2 = (1+\frac{r}{2}) Y 
 \end{align*}
However, if $R$ goes down, you can refinance and only pay $(1+r)D$ next period
 \begin{align*}
C_1 = (1-r) Y\\
C_2 =  Y
\end{align*}
If you didn't notice this, loss to utility would be \textbf{first order}.
}
\frame{
\frametitle{Costs of Inattention: A Numerical Example}
Model:
\begin{itemize}
\item 40 years of life
\item Consumption and Income constant in baseline ($\beta= 1/R$)
\item Consumer has a mortgage, face value one year of income, fixed installments for 20 years.
\item Experiment: Shock real rate - exponentially decaying shock with half life 2.5 years (5 year rates moves 0.5x size of shock
\end{itemize}
\pause
\bigskip
What are the costs of in inattention to the interest rate shock with regards to:
\begin{itemize}
\item Intertemporal Substitution
\item Mortgage Refinancing
\end{itemize}
}
\frame{
\frametitle{Costs of Inattention: Intertemporal vs Refiancing}
\centering
\includegraphics[width=\textwidth]{../../Code/HARK/Figures/CostofInattention.pdf}
}
\section{Fwd Guidance}
\frame{
\frametitle{Forward Guidance}
What is the effect of a shock to the short-term real rate 5 years in the future?\\
\pause
\bigskip
Intertemporal Substitution: Spending up for 5 years, then down thereafter. In general equilbrium $\implies$ positive output gap for 5 years $\implies$ huge inflation!\\
Effect is MUCH greater than a shock to the short-term rate today\\
\bigskip
Refinancing: Spending up today, but short-lived effect.\\
Effect is similar size (or smaller) than a shock to the short-term rate today
}
\section{NK Model}
\frame{
\frametitle{A Two-Agent NK model with Debt}
Two agents:
\begin{itemize}
\item[1] Standard unconstrained, forward-looking agent
\item[2] Hand-to-mouth agent, able to borrow, subject to borrowing constraint on income
\end{itemize}
\bigskip
Shock to Taylor Rule is VERY persistent
}
\frame{
\frametitle{Implulse Respnse Functions}
\centering
	\includegraphics[width=0.45\textwidth]{../../Code/Dolo/Figures/shock.pdf}
	\includegraphics[width=0.45\textwidth]{../../Code/Dolo/Figures/output_gap.pdf}\\
	\includegraphics[width=0.45\textwidth]{../../Code/Dolo/Figures/real_rate.pdf}
	\includegraphics[width=0.45\textwidth]{../../Code/Dolo/Figures/nominal_rate.pdf}
}
\frame{
\frametitle{Implulse Respnse Functions}
\centering
	\includegraphics[width=0.45\textwidth]{../../Code/Dolo/Figures/constrained.pdf}
	\includegraphics[width=0.45\textwidth]{../../Code/Dolo/Figures/unconstrained.pdf}\\
	\includegraphics[width=0.45\textwidth]{../../Code/Dolo/Figures/inflation.pdf}
}



\bibliographystyle{\econtexBibStyle}
\newsavebox\mytempbib
\savebox\mytempbib{\parbox{\textwidth}{\bibliography{\econtexRoot/WhoPaysAttention}}}


\end{document}


